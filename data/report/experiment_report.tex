\documentclass[sigconf]{acmart}

\AtBeginDocument{ \providecommand\BibTeX{ Bib\TeX } }
\setcopyright{acmlicensed}
\copyrightyear{2025}
\acmYear{2025}
\acmDOI{10.5281/zenodo.1234567}

\acmConference[BI 2025]{Business Intelligence}{-}{-}

\begin{document}

\title{BI2025 Experiment Report - Group 26}
%% ---Authors: Dynamically added ---

          \author{Aron Smith}
          \authornote{Student A, Matr.Nr.: 01234567}
          \affiliation{
            \institution{TU Wien}
            \country{Austria}
          }
          
          \author{Morgan Stern}
          \authornote{Student B, Matr.Nr.: 76543210}
          \affiliation{
            \institution{TU Wien}
            \country{Austria}
          }
          
          \author{َAmir Stern}
          \authornote{Student B, Matr.Nr.: 76543210}
          \affiliation{
            \institution{TU Wien}
            \country{Austria}
          }
          
          \author{Morgan Saadati}
          \authornote{Student B, Matr.Nr.: 76543210}
          \affiliation{
            \institution{TU Wien}
            \country{Austria}
          }
          
          \author{َAmir Saadati}
          \authornote{Student B, Matr.Nr.: 76543210}
          \affiliation{
            \institution{TU Wien}
            \country{Austria}
          }
          

\begin{abstract}
  This report documents the machine learning experiment for Group 26, following the CRISP-DM process model.
\end{abstract}

\ccsdesc[500]{Computing methodologies~Machine learning}
\keywords{CRISP-DM, Provenance, Knowledge Graph, Machine Learning}

\maketitle

%% --- 1. Business Understanding ---
\section{Business Understanding}

\subsection{Data Source and Scenario}
...Data source and Scenario description:
Data source: The dataset is the 'House Sales in King County, USA' from Kaggle, containing 21,613 real
house sales records in King County, Washington (Seattle area). It includes 21 attributes
: id, date, price (target), bedrooms, bathrooms, sqft\_living, sqft\_lot, floors, waterfront (binary),
view (ordinal 0-4), condition (ordinal 1-5), grade (ordinal 1-13), sqft\_above, sqft\_basement, yr\_built,
yr\_renovated, zipcode (categorical), lat,  long (continuous), sqft\_living15, sqft\_lot15. All features 
are interpretable real-estate semantics with variety: continuous numeric (sqft, lat/long), discrete numeric (bedrooms),
ordinal (condition, grade), binary (waterfront), categorical (zipcode), temporal (date, yr\_built).
No artificial/synthesized data; real public assessor records.

Scenario: A real estate agency in Seattle aims to build a predictive tool for house prices to assist clients in 
selling decisions, reducing over/under-valuation risks in a competitive market influenced by tech boom 
...

\subsection{Business Objectives}
1.Develop a robust predictive model that accurately predict house sale prices in USA
 based on property features such as  square footage, number of rooms, condition, waterfront to
support real-estate pricing decisions.

2.Analyze which property characteristics (e.g location, grade, renovations, size) 
have the most influence on sale price to help stakeholders understand market determinants.
...

%% --- 2. Data Understanding ---
\section{Data Understanding}

The following features were identified in the dataset:

\begin{table}[h]
  \caption{Raw Data Features}
  \label{tab:features}
  \begin{tabular}{lp{0.2\linewidth}p{0.4\linewidth}}
    \toprule
    \textbf{Feature Name} & \textbf{Data Type} & \textbf{Description} \\
    \midrule
    \bottomrule
  \end{tabular}
\end{table}

%% --- 3. Data Preparation ---
\section{Data Preparation}
\subsection{Data Cleaning}
Describe your Data preparation steps here and include respective graph data.


%% --- 4. Modeling ---
\section{Modeling}

\subsection{Hyperparameter Configuration}
The model was trained using the following hyperparameter settings:

\begin{table}[h]
  \caption{Hyperparameter Settings}
  \label{tab:hyperparams}
  \begin{tabular}{lp{0.4\linewidth}l}
    \toprule
    \textbf{Parameter} & \textbf{Description} & \textbf{Value} \\
    \midrule
    Learning Rate & From 0.001 to 0.1 & 1.23 \\
    \bottomrule
  \end{tabular}
\end{table}

\subsection{Training Run}
A training run was executed with the following characteristics:
\begin{itemize}
    \item \textbf{Algorithm:} Random Forest Algorithm
    \item \textbf{Start Time:} 2025-12-25 15:52:58
    \item \textbf{End Time:} 2025-12-25 15:57:51
    \item \textbf{Result:} R-squared Score = 1.2300
\end{itemize}

%% --- 5. Evaluation ---
\section{Evaluation}

%% --- 6. Deployment ---
\section{Deployment}

\section{Conclusion}

\end{document}
